\documentclass[10pt, twocolumn]{thesis}
\usepackage[utf8]{inputenc}
\usepackage{graphicx}
\usepackage{amsmath}

\title{Notes on SPH}
\author{JC}
\date{\today}

\begin{document}
\maketitle
\tableofcontents
\part{SPH for Continuoum Domain}

\chapter{Navier-Stokes equations in Lagrangian Form} 
The Navier-Stokes equations in the Langrangian description are\cite{Liu_SPH}
\begin{itemize}
\item Continuiyty equation
\[\frac{D\rho}{Dt}=-\rho\nabla\cdot v\]
This equiation is dervied from mass conservation inside a infintesimal control volume $\delta V$ in Lagrangian description( the control volume is moving in a streamline).This expression assumes that mass is conservated inside the control volume and velocity does not change across the control volume. In the equation, velocity divergence is interpreted as the time rate of volume change per unit volume.
\item Momentum en $x$ direction
\[\rho\frac{dv_x}{dt}=-\frac{\partial P}{\partial x} + \frac{\partial \tau_{xx}}{\partial x} + \frac{\partial \tau_{yx}}{\partial y} + \frac{\partial \tau_{z}x}{\partial z} + \rho F_x\]
This equation is derived from Newton second Law for Langrangian control volume assuming constant mass inside the volume. The forces in consideration here are hidrostatic Pressure $P$, body forces per volume unit $F_x$ and the forces generated by the stress state of the control volume.
\item Energy equation\\
This equation takes in account the effect of work done by isotropic pressure and the energy disipation of viscous shear forces\\
\end{itemize}
Those equations can be expressed in a more compact way. Superscripts denote coordinate directions. The summation of the equations is taken over repeated indices and the total time derivatives are taken in the moving Langrangian frame.
\begin{itemize}
\item Continuity
\[\frac{D\rho}{\Delta t} = -\rho\frac{\partial v^\beta}{\partial x^\beta}\]
\item Momentum
\[\frac{Dv^\alpha}{\Delta t} = \frac{1}{\rho}\frac{\partial\sigma^{\alpha\beta}}{\partial x^{\beta}}\]
\item Energy
\[\frac{De}{Dt}=\frac{1}{\rho}\sigma^{\alpha\beta}\frac{\partial v^\alpha}{\partial x^\beta}\]
\end{itemize}
Those equations use the stress tensor $\sigma^{\alpha\beta}$ as a combination of Pressure $P$ and the stresess supported by the control volume $\tau_{ij}$.
\[\sigma^{\alpha\beta} = -P\delta^{\alpha\beta} + \tau^{\alpha\beta}\]
In the case of fluid dynamics, the deviatoric stress is given by:
\[\tau^{\alpha\beta}=\mu\epsilon^{\alpha\beta}\]
but, in the case of solid mechanics, the deviatoric stress is proportional to the strain rate tensor\cite{Luna_SPH}\cite{allahdadi93} 
\begin{scriptsize}
There is a little difference with the equation proposed by \cite{benz95}. The equation on \cite{allahdadi93} is difficult to see because of the scanning conditions
\end{scriptsize}
\[\dot{\tau}^{\alpha\beta}=\mu\bar{\epsilon^{\alpha\beta}} = \mu\left(\dot{\epsilon}^{\alpha\beta}-\frac{1}{3}\delta^{\alpha\beta}\dot{\epsilon}^{\gamma\gamma}\right)\]
where $\mu$ is the shear modulus and $\epsilon$ is the traceless rate fo strain.
Rotation terms are needed to allow the transformation of the stress from the reference frame associated with the material to the laboratory reference frame wich all other equations are specified\cite{allahdadi93}\cite{benz95}. The Jaumann rate is the most widely used in codes and we adop it also. With Jauman rate our constitutive equation is
\[\dot{\tau}^{\alpha\beta} = \mu\left(\dot{\epsilon}^{\alpha\beta}-\frac{1}{3}\delta^{\alpha\beta}\dot{\epsilon}^{\gamma\gamma}\right) + \tau^{\alpha\gamma}R^{\beta\gamma} + \tau^{\beta\gamma}R^{\alpha\gamma}\]
\begin{scriptsize}
There are some missundesrtandings about the superindexes of $R$ terms)
\end{scriptsize}
The strain rate and rotation rate that have been used are defined as follows\cite{allahdadi93}\cite{benz95}
\[\dot{\epsilon}^{\alpha\beta}=\frac{1}{2}\left(\frac{\partial v^\alpha}{\partial x^\beta} + \frac{\partial v^\beta}{\partial x^\alpha}\right)\]
\[R^{\alpha\beta}=\frac{1}{2}\left(\frac{\partial v^\alpha}{\partial x^\beta} - \frac{\partial v^\beta}{\partial x^\alpha}\right)\]
The calculations for each part of $\sigma$ dependes on the equiation of state (for pressure) and the material model (for stress).
The 'plastic behaviour' can be introduced in the equations using the von Mises yielding criterion. We limit our deviatoric stress tensor by\cite{benz95}
\[S^{\alpha\beta}\Rightarrow fS^{\alpha\beta}\]
where $f$ is computed from
\[f=min\left(\frac{Y_o^2}{3J_2},1\right)\]
and $J_2$ is the second invariant od the deviatoric stress tensro defined as
\[J_2=\frac{1}{2}S^{\alpha\beta}S^{\alpha\beta}\]

\begin{scriptsize}
In reference \cite{allahdadi93} the model is similar but there some little differences
\textit{***I think this could be an important topic to achieve my objectives in the SPH project***}
\end{scriptsize}
\chapter{Material Model - Fracture}


\part{Mathematical Review}
\chapter{Calculus}
\textbf{Vector}: The term \textbf{vector} is used by scientist to indicate a quantity (such as displacement or velocity or force) that has both \textbf{magnitude} and \textbf{direction}\cite{S_calculus}
\section{Dot Product}
\textbf{Definition} If a = $vec{a}=\langle b_1,a_2,a_3\rangle$ and $vec{b}=\langle b_1,b_2,b_3\rangle$, then the dot product of $a$ and $b$ is the number $a\cdot b$ given by
\[\vec{a}\cdot\vec{b}=a_1b_1+a_2b_2+a_3b_3\]
If $\theta$ is the angle between the vector $\vec{a}$ and $\vec{b}$ then
\[\vec{a}\cdot\vec{b}=|a||b|\cos{\theta}\]
Note that if $\vec{a}$ and $\vec{b}$ are parallel vectors, then $\theta=[0, \pi]$ and $\cos{\theta}=1$.\\
Two vector $\vec{a}$ and $\vec{b}$ are orthogonal if and only if $\vec{a}\cdot\vec{b}=0$, which means that $\theta=\pi/2$.
\section{Divergence}
If $F=P\hat{\imath}+Q\hat{\jmath}+R\hat{k}$ is a vector Field in $R^3$, then the \textbf{divergence of $F$} is the function of three variables defined by (assuming that the partial derivates exists):
\[div(F)=\nabla\cdot F=\frac{\partial P}{\partial x}+\frac{\partial Q}{\partial y}+\frac{\partial R}{\partial z}\]
observe that $\nabla\cdot F$ is a scalar field.\\
If $F$ is a vector field on $R^3$, then $curl(F)$ is also a vector field on $R^3$. As such, we can compute its divergence:
\[div\,curl(F)=0\]
\section{Divergence Theorem}
Let $E$ be a simple solid region and let $S$ be the boundary surface of $E$, given with positive (outward) orientation. Let $F$ be a vector field whose component functions have continuos partial derivatives on an open region that contains $E$. Then
\[\iint_{S}F\cdot dS = \iiint_{E}\nabla\cdot FdV\]
The divergence theorem states that, under the given conditions, the flux of $F$ across the boundary surface of $E$ is equal to the volume integral of the divergence of $F$ over $E$ i.e the theorem let us change a surface integral to a volume integral.
\bibliography{mybib}
\bibliographystyle{unsrt}
\end{document}